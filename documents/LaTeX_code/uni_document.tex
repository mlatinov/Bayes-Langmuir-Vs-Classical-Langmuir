% !TeX program = xelatex
\documentclass[12pt]{article}
\usepackage{fontspec}
\setmainfont{Times New Roman}
\usepackage{geometry}
\geometry{margin=1in}
\usepackage{amsmath, amssymb}
\usepackage{graphicx}
\usepackage{booktabs}
\usepackage{graphicx}
%\usepackage{hyperref}  % Закоментирано временно
\setlength{\parindent}{1em}
\setlength{\parskip}{0.8em}
\setlength{\parindent}{0.5cm}

\title{Лабораторен протокол: Изолиране, количествено определяне и качествен анализ на фосфолипиди от белодробна тъкан}
\author{Методи Латинов , Илонка Георгиева , Виктория Тенева}
\date{\today}
\begin{document}
	\maketitle

% Abstract 
\section{Абстракт}
\hspace{0.5cm}Белодробният сърфактант представлява повърхностноактивно вещество (ПАВ), есенциално за живота на организмите, използващи бели дробове като основен дихателен орган. Той покрива вътрешната стена на алвеолите. Секретира се от пневмоцити тип II, които съставят около 10-12\% от алвеоларният епител. Обикновено е под форма на мултислой, изграден от различни фосфолипиди, протеини и йони. 

Белодробния сърфактант предотвратява алвеоларния колапс при процесът на издишване (т.е слепването на алвеолите), чрез своята роля на ПАВ, намалява повърхностното напрежение, формиращо се на границата вода-въздух в алвеолите. Дефицит на белодробен сърфактант води до  развитие на Неонатален респираторен дистрес синдром (Infant respiratory distress syndrome, мед.) при новородени. 

В медицината основно се използва Surfaxin (второ поколение синтетичен сърфактант) като екзогенен сърфактантен аналог, който съдържа синтетичен пептид (KL-4), сходен по функция с естественият сърфактантен протеин SP-B.  Все още обаче не е изяснена причината за проявата на сърфактантният дефицит, което от своя страна стимулира изследване и анализ на състава, структурата, биохимичната синтеза и физиологична регулация на белодробният сърфактант.
% Introduction ---------------------------------------
\section{Въведение} 

% Methodology ---------------------------------------------
\section{Методология}

% From the First lab
\subsection{Изолиране на фосфолипиди като компоненти на белодробния сърфактант от белодробна тъкан}

\hspace{0.5cm}Работата се извършва с прясно изолирани тъканни проби с цел запазване на целостта и биологичната активност на клетките и техните компоненти, по-специално на фосфолипидите, които са основна съставка на белодробния сърфактант.

\subsubsection{Условия на работа}
\begin{itemize}
	\item Всички етапи от процедурата се извършват \textbf{при ниски температури} (работа на лед).
	\item Основната причина за това е \textbf{инхибирането на ензимната активност} на фосфолипази, които в противен случай биха катализирали разграждането на целевите фосфолипиди.
	\item Спазването на този протокол е от съществено значение за \textbf{максимизиране на добива} на интактни липидни молекули.
\end{itemize}

\subsubsection{Екстракция на липиди}
\hspace{0.5cm}За екстракцията на липидите от хомогенизираната тъкан се използва метод, базиран на тяхното свойство да са \textbf{неразтворими във вода и разтворими в неполярни органични разтворители}.
\begin{itemize}
	\item Като екстракционна смес се използва комбинация от \textbf{хлороформ и метанол}.
	\item При добавянето на тази смес към хомогената, съдържащите се липиди преминават в органичната фаза, докато всички водноразтворими компоненти (протеини, захари, соли) остават във водната фаза.
\end{itemize}

\subsubsection{Материал на съдове}
Поради агресивността на използваните разтворители се налага спазване на строги мерки:
\begin{itemize}
	\item Всички манипулации с хлороформ се извършват \textbf{изключително в стъклени лабораторни съдове}.
	\item Хлороформът катализира разграждането на пластмаси, което води до:
	\begin{enumerate}
		\item \textbf{Увреждане на лабораторния консуматив}.
		\item \textbf{Замърсяване на пробата} с частици от разградената пластмаса.
	\end{enumerate}
\end{itemize}

\subsubsection{Съхранение на липидния екстракт}
\hspace{0.5cm}Полученият липиден екстракт не подлежи на дълго съхранение при стандартни условия поради висок риск от окислително увреждане.
\begin{itemize}
	\item Липидите в органични разтворители са особено податливи на \textbf{прекисно окисление}, тъй като разтворителите пропускат кислород по-лесно от водна среда.
	\item Кислородът инциира образуването на \textbf{свободни радикали}, които влизат във верижна реакция с липидните молекули.
	\item Тази радикална верижна реакция води до \textbf{масово разграждане} на липидите в пробата.
	\item За предотвратяване на този процес, екстрактът се \textbf{съхранява при ниски температури (замразяване)} след като бъде получен.
\end{itemize}

\subsection{Практическо изпълнение}

\subsubsection{Подготовка на тъканта}
\begin{enumerate}
	\item Подготвихме ледена банка, като запълнихме дъното на петриев съд с лед.
	\item С помощта на стерилни ножица и пинцет отрязахме фрагмент от белодробна тъкан с маса \textbf{0,96 g}. Тъканта беше поставена върху алуминиево фолио в петриевия съд.
	\item Накълцахме тъканта на дребно директно върху фолиото на ледената банка.
\end{enumerate}

\subsubsection{Приготвяне на хомогенат}
\begin{enumerate}
	\item Приготвихме физиологичен разтвор (0,9\% NaCl) в съотношение \textbf{10 mL разтвор за 10 g тъкан}.
	\item Изчислихме необходим обем: 
	\[
	\frac{0,96\text{ g тъкан} \times 10\text{ mL разтвор}}{10\text{ g тъкан}} = 9,6\text{ mL физиологичен разтвор}
	\]
	\item Отмерихме \textbf{9,6 mL} физиологичен разтвор в мерителен цилиндър и го охладихме на лед.
	\item Накълцаната тъкан беше прехвърлена в специална епруветка за хомогенизиране.
	\item Добавихме приблизително $\frac{1}{3}$ от физиологичния разтвор и хомогенизирахме механично чрез последователно натискане, усукване, вдигане и въртене на пестика.
	\item Хомогенатът беше изсипан в центрофужна епруветка.
	\item Процедурата беше повторена още два пъти с останалите порции разтвор.
	\item Общата маса на хомогената беше измерена: \textbf{14,55 g}
	\item За контрол измерихме тара на празна центрофужна епруветка с вода: \textbf{14,58 g}
\end{enumerate}

\subsubsection{Центрофугиране}
\begin{enumerate}
	\item Центрофужната епруветка с хомогената беше поставена в центрофуга с въртящ ротор.
	\item Центрофугирахме при \textbf{1600 об/мин} за \textbf{15 минути}.
	\item След центрофугиране, супернатантът беше отделен и прехвърлен в мерителна епруветка.
	\item Получихме \textbf{9,2 mL} супернатант, който беше поставен на лед.
\end{enumerate}

\subsubsection{Екстракция на липиди}
\textbf{Изчисление на необходимите обеми разтворители:}

Имаме целево съотношение на разтворителите: \textbf{5:5:4} (метанол:хлороформ:водна фаза)

\begin{itemize}
	\item Супернатант: 9,2 mL
	\item Допълнителен физиологичен разтвор: 9,2 mL  
	\item \textbf{Обща водна фаза:} 9,2 + 9,2 = \textbf{18,4 mL}
\end{itemize}

\begin{align*}
	\text{1 част} &= \frac{18,4\text{ mL}}{4} = 4,6\text{ mL} \\
	\text{5 части метанол} &= 5 \times 4,6 = 23,0\text{ mL} \\
	\text{5 части хлороформ} &= 5 \times 4,6 = 23,0\text{ mL}
\end{align*}

Практически закръглихме до: \textbf{24 mL метанол} и \textbf{12 mL хлороформ} (в 2:1 съотношение).

\subsubsection{Процедура на екстракция}
\begin{enumerate}
	\item Приготвихме смес от \textbf{24 mL метанол} и \textbf{12 mL хлороформ} в стъклено шише.
	\item Прехвърлихме супернатанта и разтворителната смес в делителна фуния.
	\item Енергично разклащахме фунията за \textbf{30 секунди}, като периодично я обръщахме и отваряхме кранчето за освобождаване на налягането.
	\item Добавихме \textbf{11 mL} чист хлороформ и продължихме с разклащането.
	\item Добавихме \textbf{9,2 mL} физиологичен разтвор и разклащахме още.
	\item Оставихме фунията на статива за разделяне на фазите.
\end{enumerate}

\subsubsection{Разделяне на фази и събиране на екстракта}
\begin{itemize}
	\item Наблюдавахме образуването на две ясно различими фази:
	\begin{itemize}
		\item \textbf{Долна фаза:} хлороформна (по-тежка), съдържаща липидите
		\item \textbf{Горна фаза:} водно-метанолна
		\item \textbf{Интерфаза:} белтъчни агрегати
	\end{itemize}
	\item Внимателно отделихме долната хлороформна фаза в чиста крушовидна колба.
	\item Приготвихме аликвоти за следващо упражнение:
	\begin{itemize}
		\item 3 стъклени епруветки: \textbf{75 μL, 150 μL, 300 μL}
		\item 2 епендорфки: \textbf{500 μL} всяка
	\end{itemize}
	\item Епруветките бяха оставени отворени за изпаряване на разтворителя.
\end{itemize}

% From the Second Lab
\subsection{Определяне на количество фисфолипиди от белодробната тъкан}
\hspace{0.5cm}Количественото определяне на фосфолипиди се осъществява спектрофотометрично. Идеята е да получим цветна реакция. В случая ще определим концентрацията на фосфор в съответния екстракт от белодробен сърфактант. Използваме фосфорната концентрация, тъй като фосфорът е пряко свързан с фосфолипидната структура. 

За да определим количеството му в екстракта, е необходимо да използваме определен реактив, чрез който да постигнем разкъсване на връзката между фосфора и структурата на фосфолипидите в пробата. Реактивът, който използваме, е фосфолипаза C и D, тъй като фосфолипаза C разкъсва хидрофилната глава изцяло, а фосфолипаза D разкъсва връзката със страничната R-група.

Подходът, който използваме, е да действаме със силен окислител и загряване. Използваме факта, че органичните съединения в среда със силен окислител и при висока температура се превръщат в неорганични съединения (CO$_2$) чрез процеса минерализация. Поради това избираме да приложим такива условия към екстракта от бял дроб. Очакваният резултат е получаване на неорганичен фосфат и други неорганични продукти.

Използваме колориметричен метод за определяне на количеството фосфолипиди на база тяхното съдържание на фосфор. Този метод е много подобен на този на Кофехкова и Девич, като е важно да отбележим, че за цветната реакция използваме реактив на Хан (Hahn). Реактивът представлява алкохолен разтвор на 8-хидроксихинолин. Обикновено се използва за идентификация на минерали и утаяване на тежки метали.

\subsubsection{Практическо изпълнение}

\begin{enumerate}
	\item Към трите проби от липидния екстракт (обеми 75 μL, 150 μL и 300 μL) се добавят по \textbf{50 μL 50\% H\textsubscript{2}SO\textsubscript{4}}.
	\item Пробите се вортексират и се загряват при \textbf{200°C за 15 минути} за минерализация.
	\item Приготвя се стандартна проба (СТ) от \textbf{1 mL} стандартен разтвор с известна концентрация на фосфор.
	\item Към всички проби (включително стандарт и контрол) се добавят по \textbf{1 mL реактив на Хан}.
	\item Добавят се по \textbf{4 mL вода} във всички проби (с изключение на стандартната).
	\item Пробите се вортексират и поставят на водна баня.
	\\item След охлаждане, абсорбцията на пробите се измерва на спектрофотометър при \textbf{700 nm}.
\end{enumerate}

% From the 3th Lab
\subsection{Качествен анализ на белодробния сърфактант}

\hspace{0.5cm}Тънкослойната хроматография (ТХ) е планарна хроматографична техника, при която разделянето на липидните компоненти се основава на тяхното разпределение между неподвижна фаза (силикагел) и подвижна фаза (органичен разтворител). Подвижната фаза има основно значение за получения липиден профил, като всяка композиция дава различно разделяне на компонентите.

\subsubsection{Подготовка на плаките и пробите}
\begin{itemize}
	\item \textbf{Активиране на плаките:} Плаките с силикагел се активират чрез инкубиране при температура между \textbf{90-110°C} за \textbf{1 час} с цел отстраняване на адсорбираната влага.
	\item \textbf{Приготвяне на пробата:} В епендорфка се приготвя разтвор от \textbf{80 μL} хлороформ и липидния екстракт. Съдът се накланя и върти за равномерно разтваряне.
	\item \textbf{Промиване на капиляра:} Капилярната пръчица се промива с чист хлороформ преди всяко използване.
\end{itemize}

\subsubsection{Нанасяне на пробите и развитие на хроматограмата}
\begin{enumerate}
	\item Стартовите точки се разполагат на \textbf{2 cm} от долния край на плаката, с разстояние \textbf{0,5 cm} от страничните краища и между отделните проби.
	\item Пробата се нанася \textbf{капка по капка} чрез многократно допиране на капиляра, като се образува чертичка от 4-5 точки.
	\item Процесът се повтаря с допълнителни \textbf{40 μL} разтворител за концентриране на пробата.
	\item Като стандарт се нанася \textbf{20 μL} разтвор на фосфатидилетаноламин (1 mg/mL).
	\item Плаката се поставя в хроматографски съд, съдържащ подвижна фаза със следния състав:
	\begin{center}
		\textbf{30:9:25:6:18 = CHCl\textsubscript{3}:CH\textsubscript{3}OH:iPrOH:0.25\% KCl:3-етиламин}
	\end{center}
	\item Вътрешната повърхност на съда е облепена с филтърна хартия, която се напоява с разтворители и предотвратява изпарението.
	\item Хроматографията продължава докато фронтът на разтворителя достигне до \textbf{2 cm} от горния край на плаката.
\end{enumerate}

\subsubsection{Проявяване и детекция}
\begin{itemize}
	\item Използва се \textbf{твърдо проявяване} с необратими нарушения на липидите.
	\item След разделянето, плаките се изсушават и пръскат с \textbf{20\% водно-метанолов разтвор на амониев сулфат}.
	\item Плаките се поставят при \textbf{200°C} за \textbf{10 минути} за минерализация.
	\item Определя се \textbf{R\textsubscript{f}} стойността - относителното разстояние, което определен липид изминава от стартовата точка.
\end{itemize}

Чакам снимка Йони ?????????????

% From the 4th Lab
\subsection{Определяне на молекулна площ от изотерми на Лангмюър}

\hspace{0.5cm}Лангмюировата везна представлява система за изследване на мономолекулни слоеве на повърхността на вода. В лабораторна ваничка се налива воден разтвор, така че повърхността на водата да е малко над стената на тавичката, като се формира менискус. Липидите се нанасят по повърхността на водния слой. 

Основната идея на Лангмюировата везна е да компресира посредством прегради двустранно водната фаза с повърхностно разположени липиди. Както преградите, така и тавичката са с тефлонови покрития, което предотвратява прилипане на липидите.

В средната част на тавичката, потопена във водната фаза с липиди, се разполага игла от благороден метал. Тя изпълнява ролята на датчик, който регистрира промяната в повърхностното напрежение на липидния монослой и промяната на фазовите състояния на липидите. 

\subsubsection{Приготвяне на пробата}
\begin{enumerate}
	\item Към епендорфка със запазения липиден екстракт се добавят \textbf{500 μL} хлороформ.
	\item Изчислява се обемът на екстракта, съдържащ \textbf{6 μg} липиди:
	
	\textbf{Изчисление:}
	\begin{align*}
		\text{Концентрация: } & 186 \text{ mg/mL} = 186 \text{ μg/μL} \\
		\text{Необходим обем: } & V = \frac{6 \text{ μg}}{186 \text{ μg/μL}} = 0,03226 \text{ μL} \approx 32,26 \text{ nL}
	\end{align*}
	
	\item Практически се приготвя по-голямо разреждане: \textbf{32,60 μL} от разредения екстракт се пипетират внимателно.
\end{enumerate}

\subsubsection{Нанасяне на пробата в Лангмюировата везна}
\begin{enumerate}
	\item Подготвената проба се нанася \textbf{капка по капка} на различни места по водната повърхност в тенка на Лангмюър.
	\item Хлороформът се изпарява бързо, оставяйки мономолекулния слой от липиди.
	\item След изпаряване на разтворителя, бариерата започва да компресира monolayers-а.
\end{enumerate}

% Results ------------------------------------------
\section{Резултати}

\subsection{Резултати от количествен анализ}
\subsubsection{Изчисления}

\textbf{1. Определяне на концентрацията на фосфор в пробите:}

За стандартна проба с абсорбция 0,698 съответства на 10 μg/mL фосфор.

Изчисляваме концентрацията на фосфор за всяка проба по формулата:
\[
[P]_{\text{проба}} = \frac{A_{\text{проба}}}{A_{\text{стандарт}}} \times 10 \text{ μg/mL}
\]

\begin{itemize}
	\item За 75 μL: $[P] = \frac{0,039}{0,698} \times 10 = 0,56$ μg/mL
	\item За 150 μL: $[P] = \frac{0,079}{0,698} \times 10 = 1,13$ μg/mL  
	\item За 300 μL: $[P] = \frac{0,149}{0,698} \times 10 = 2,13$ μg/mL
\end{itemize}

\textbf{2. Преизчисляване за 1 mL обем:}

Тъй като пробите са с различен обем, преизчисляваме концентрацията за 1 mL (1000 μL):

\begin{itemize}
	\item За 75 μL: $[P]_{1mL} = \frac{0,56 \times 1000}{75} = 7,47$ μg/mL
	\item За 150 μL: $[P]_{1mL} = \frac{1,13 \times 1000}{150} = 7,53$ μg/mL
	\item За 300 μL: $[P]_{1mL} = \frac{2,13 \times 1000}{300} = 7,10$ μg/mL
\end{itemize}

\textbf{3. Осреднена стойност:}
\[
[P]_{\text{средно}} = \frac{7,47 + 7,53 + 7,10}{3} = 7,37 \text{ μg/mL}
\]

\textbf{4. Изчисляване на фосфолипиди:}

Приемайки, че средното съдържание на фосфор във фосфолипидите е 4\%, изчисляваме съдържанието на фосфолипиди:

\[
[PL] = \frac{[P]}{0,04} = \frac{7,37}{0,04} = 184,25 \text{ μg/mL}
\]

Чрез спектрофотометричен метод беше определено средното съдържание на фосфолипиди в екстракта: \textbf{184,25 μg/mL}.

\subsubsection{Експериментални резултати}
\vspace{-0.5cm}
\begin{table}[htbp]
	\centering
	\caption{Резултати от количественото определяне на фосфолипиди}
	\begin{tabular}{|c|c|c|c|c|c|}
		\hline
		Обем на пробата & Абсорбция & [P] проба & [P] 1ml & [PL] средно & [PL] \\
		& (700 nm) & μg/mL & μg/mL & μg/mL & μg/ml \\
		\hline
		75 μL & 0,039 & 0,56 & 7,46 & - & - \\
		\hline
		150 μL & 0,079 & 1,13 & 7,53 & 7,36 & 184 \\
		\hline
		300 μL & 0,149 & 2,13 & 7,10 & - & - \\
		\hline
		Стандарт & 0,698 & 10,00 & - & - & - \\
		\hline
	\end{tabular}
\end{table}

\subsection{Резултати от качествен анализ }
Чакам Йони снимка ?????????????

\subsection{Резултати от изотерми на Лангмюър}

\begin{figure}[h]
	\centering
	\includegraphics[width=0.9\textwidth]{C:/Users/Huawei/OneDrive/Bayes_analysis/Bayes_Langmuir_Vs_Classical_Langmuir/Bayes-Langmuir-Vs-Classical-Langmuir/plots/Compression_isotherm_plot.png}
	\caption{Компресионна изотерма на липиден екстракт от белодробна тъкан. Червен триъгълник маркира точката на максимално повърхностно налягане, използвана за изчисляване на молекулната площ.}
	\label{fig:isotherm}
\end{figure}
\vspace{0.5cm}
\subsubsection{Изчисляване на площта на молекулата}

\textbf{1. Определяне на минималната площ от изотермата:}
\begin{equation}
	A_{\text{min}} = A(\pi_{\text{max}})
\end{equation}
където $\pi_{\text{max}}$ е максималното измерено повърхностно налягане.

\textbf{2. Изчисляване на броя молекули:}
\begin{align}
	n &= \frac{m}{M} \\
	N &= n \times N_A
\end{align}
\vspace{-0.5em}
\begin{flushleft}
	\small{където: \\
		$m = 6 \times 10^{-6}$ g - маса на нанесените липиди \\
		$M = 750$ g/mol - молекулна маса \\
		$n$ - количество вещество в молове \\
		$N_A = 6.022 \times 10^{23}$ mol$^{-1}$ - число на Авогадро \\
		$N$ - общ брой молекули}
\end{flushleft}

\textbf{3. Преобразуване на единици и изчисляване на молекулната площ:}
\begin{align}
	A_{\text{min}}^{\text{Å}^2} &= A_{\text{min}}^{\text{mm}^2} \times 10^{14} \\
	A_{\text{mol}} &= \frac{A_{\text{min}}^{\text{Å}^2}}{N}
\end{align}
\vspace{-0.5em}
\begin{flushleft}
	\small{където: \\
		$10^{14}$ е преобразувателният фактор ($1$ mm$^2 = 10^{14}$ Å$^2$) \\
		$A_{\text{mol}}$ - площ на една молекула в Å$^2$}
\end{flushleft}

\subsubsection{Експериментални стойности}
\begin{itemize}
	\item Минимална площ при $\pi_{\text{max}}$: $A_{\text{min}} = 3786.585$ mm$^2$
	\item Брой молекули в монослоя: $N = 4.817 \times 10^{15}$
	\item Получена молекулна площ: $A_{\text{mol}} = 78$ Å$^2$
\end{itemize}
\vspace{4cm}
% Open Section --------------------------------------
\section{Заключение}

\end{document}

