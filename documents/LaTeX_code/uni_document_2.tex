% !TeX program = xelatex
\documentclass[12pt]{article}
\usepackage{fontspec}
\setmainfont{Times New Roman}
\usepackage{geometry}
\geometry{margin=1in}
\usepackage{amsmath, amssymb}
\usepackage{graphicx}
\usepackage{booktabs}
%\usepackage{hyperref}  % Закоментирано временно
\setlength{\parindent}{1em}
\setlength{\parskip}{0.8em}

\title{Сравнителен анализ: Класически vs Бейзилиански подход за определяне на молекулна площ от изотерми на Лангмюър}

\author{Методи Латинов}

\date{\today}

\begin{document}
	\maketitle

\section{Абстракт}
\section{Въведение : Мотивация и Аргументи}
\section{Класически подход}
\section{Бейзилиански подход}
\section{Бейзилиански модел}
Моделът приема,че наблюдаваните стойности на повърхностното налягане идват от нормално разпределение около теоретичната стойност,зададена от Лангмюровото уравнение

$$\pi_i \sim\mathcal{N}(\mu_i,\sigma^2)$$

където $\pi_i$ е измереното повърхностно налягане при дадена площ $A_i$,а $\mu_i$ е очакваната стойност е дефинирана като :

$$\mu_i =\pi_{\max}\left(1-
\frac{A_{\text{mol}}}
{A_{\text{mol}}+ A_i}\right)$$

Като в това уравнение $\pi_{\max}$ е максималното повърхностно налягане,
което системата достига при насищане

$A_\text{mol}$ е характерната молекулна площ,която описва степента на запълване на повърността

$\sigma$ обозначава стандартното отклонение от измерването 

Като в модела третираме $\pi_{\max}$ , $A_\text{mol}$ ,$\sigma$ като случайни величини идващи от предварително избрани разпределения.Постериорно разпределение се дефинира като :

$$
p(\theta\mid\pi,A)\propto
p(\pi\mid A,\theta)\,
p(\theta)
$$
където $p(\pi\mid A,\theta)$ е likelihood на данните ,а $p(\theta)$ е съвкупността от приорни разпределения

В модела използваните приори са следните :
$$\pi_{\max} \sim \text{Normal}(50,20)$$
$$A_\text{mol} \sim\text{Normal}(10,5)$$
$$\sigma \sim\text{Exponential}(1)$$

Предположенията са че максималното повърхностно налягане е около 50 с широка вариация.Характерната молекулна площ е около 10 и стандартното отклонение е положително и вероятно малко 


Алтернативно модела може да се изрази по следния начин : 

$$\pi_i \sim \mathcal{N}\left(
\pi_{\text{max}} \times \left(1 - \frac{A_{\text{mol}}}{A_{\text{mol}} + A_i}\right),
\sigma^2
\right)$$

Постериорните разпределения на параметрите бяха изчислени чрез Hamiltonial Monte Carlo (HMC).Използвани бяха 4 паралелни вериги с 4000 итерации,за да се осигури конвергенция и числова стабилност на оценките

\section{Диагностика на модела}
\section{Анализ и интерпретация на резултатите}
\section{Сравнение на резултатите с Класическия подход}
\section{Заключение}

\end{document}